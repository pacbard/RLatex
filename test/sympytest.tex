% Copyright (C) 2009-2011 by Tim Molteno <tim@physics.otago.ac.nz>
%
% This file may be redistributed and/or
% modified under the terms of the GNU General Public License as
% published by the Free Software Foundation, either version 2 of the
% License, or (at your option) any later version. To view a copy of this
% license, see http://www.gnu.org/licenses/ or send a letter to
% the Free Software Foundation, Inc., 51 Franklin Street, Fifth Floor,
% Boston, MA 02110-1301, USA.

%rubber: onchange sympytest.sympy "python sympytest.sympy && pdflatex sympytest.tex"

\documentclass{article}
\usepackage{amsmath}
\usepackage{amsfonts}
\usepackage{amssymb}
\usepackage{sympytex}

\title{SympyTeX example}
\author{Tim Molteno\thanks{tim@physics.otago.ac.nz}}


\begin{document}

\maketitle

\section{Important! Setting up Your \LaTeX environment}

To use SympyTeX, you have to run latex (or pdflatex) on your file, then python, and then latex again. 
\begin{verbatim}
 latex example.tex
 python example.sympy
 latex example.tex
\end{verbatim}
If you use Kile, or another \LaTeX environment, you can set it up to automatically do the
sympy python command as part of your build process.

\section{Introduction}

Using sympy within your \LaTeX document is as easy as $2+2=\sympy{2+2}$. 

You can write a block, and then use the variables defined later in your code!
\begin{sympyblock}
x = sympy.Symbol('x')
h = sympy.integrate(1+x,x)
\end{sympyblock}
The variable $h$, how can be called using {\verb \sympy{h} }, and you will get $h = \sympy{h}$. Similarly, the integral of $1+x^4$ is $\sympy{sympy.integrate(1+x**4,x)}$.

\section{sympyplain}

The sympyplain command places the sympy output as text (rather than LaTeX formatted math stuff). This is really useful if the output is huge as formulae don't line wrap.

The integral of $1+x^4$ is also \sympyplain{sympy.integrate(1+x**4,x)}.

\begin{sympyblock}
from sympy import *
e = 1/cos(x)
\end{sympyblock}
The series expansion is $\sympy{e.series(x, 0, 10)}$.

\subsection{Invisibly sympy code}

Placing a {\tt sympysilent} block in your code, allows you do do things like define
symbols without making a mess.
\begin{sympysilent}
theta = sympy.Symbol('theta')
from sympy import *
def RotationMatrixZ(angle):
  return Matrix([[cos(angle), -sin(angle), 0],[sin(angle), cos(angle),0],[0,0,1]])
\end{sympysilent}
The rotation matrix is through an angle $\theta$ around the $z$-axis is
\[ R(\theta) = \sympy{RotationMatrixZ(theta)}. \]

\end{document}

